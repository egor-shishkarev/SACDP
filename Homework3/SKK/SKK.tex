\documentclass{article}
\usepackage[russian]{babel}
\usepackage{amsmath}
\usepackage{geometry} \geometry{verbose,a4paper,tmargin=2cm,bmargin=2cm,lmargin=2.5cm,rmargin=1.5cm}

\title{\huge SKK}
\author{\LARGEШишкарев Егор}
\date{\large28 Марта 2024}

\begin{document}
\fontsize{14}{14}\selectfont
\maketitle

\section{Доказательство равенства}
Докажем, что S K K = I в терминах лямба-выражений.
Вспомним, что означает каждый комбинатор
\begin{equation}
    S = \lambda x~y~z.x~z~(y~z)
\end{equation}
\begin{equation}
    K = \lambda x~y.x
\end{equation}
\begin{equation}
    I = \lambda x.x
\end{equation}
Вычислим S K K и покажем, что результат равен I
\begin{equation}
    S~K~K = (\lambda x~y~z.x~z~(y~z))~K~K
\end{equation}
\begin{equation}
    \rightarrow_\beta~\lambda y~z.K~z~(y~z)~K    
\end{equation}
\begin{equation}
    \rightarrow_\beta~\lambda z.K~z~(K~z)  
\end{equation}
Вместо K подставляем лямбда-выражение
\begin{equation}
    = \lambda z.(\lambda x~y.x~z~(\lambda x~y.x~z))
\end{equation}
\begin{equation}
    \rightarrow_\beta~\lambda z.(\lambda y.z~(\lambda x~y.x~z))
\end{equation}
\begin{equation}
    \rightarrow_\beta~\lambda z.(z)
\end{equation}
\begin{equation}
    \rightarrow_\alpha~\lambda x.x = I
\end{equation}
Что и требовалось доказать
\end{document}

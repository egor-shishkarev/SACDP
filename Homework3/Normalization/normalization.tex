\documentclass{article}
\usepackage[russian]{babel}
\usepackage{amsmath}
\usepackage{geometry} \geometry{verbose,a4paper,tmargin=2cm,bmargin=2cm,lmargin=2.5cm,rmargin=1.5cm}

\title{\hugeЛямбда-выражение}
\author{\LARGEШишкарев Егор}
\date{\large28 Марта 2024}

\begin{document}
\maketitle
\section{Нормализация выражения}
\fontsize{14}{14}\selectfont
\begin{equation}
    \Bigl(\bigl(\lambda a.(\lambda b.b~b)~(\lambda b.b~b)\bigr)~b\Bigr)~\Bigl(\bigl(\lambda c.(c~b)\bigr)~(\lambda a.a)\Bigr)  
\end{equation}
Преобразовываем первую скобку
\begin{equation}
    \rightarrow_\beta~
    \bigl((\lambda b.b~b)~(\lambda b.b~b)\bigr)~\Bigl(\bigl(\lambda c.(c~b)\bigr)~(\lambda a.a)\Bigr) 
\end{equation}
Как мы знаем выражение в первых скобках при нормальной стратегии редукции не будет меняться. Перейдем ко второй скобке
\begin{equation}
    \rightarrow_\beta~
    \bigl((\lambda b.b~b)~(\lambda b.b~b)\bigr)~\bigl(\lambda a.a~b\bigr) 
\end{equation}
\begin{equation}
    \rightarrow_\beta~
    \bigl((\lambda b.b~b)~(\lambda b.b~b)\bigr)~b
\end{equation}
На этом этапе попытка применить $\beta$-редукцию будет зацикливаться, мы будем получать то же самое выражение. Следовательно, это лямбда-выражение не имеет нормальной формы, ведь если бы она была, то с помощью $\beta$-редукций и нормальной стратегии мы бы пришли к ней.
\end{document}
